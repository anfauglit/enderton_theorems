\documentclass[letterpaper, 10pt]{article}
\usepackage{amsmath, amsthm, amssymb}
\usepackage{mathtools}
\usepackage{enumitem}
\usepackage{listings}
\usepackage{mathrsfs}

\newtheorem{thm}{Theorem}[section]
\numberwithin{thm}{section}
\newtheorem{axiom}[thm]{Axiom}
\newtheorem{lem}[thm]{Lemma}
\newtheorem{prop}[thm]{Proposition}
\newtheorem{col}[thm]{Corollary}

\theoremstyle{definition}
\newtheorem{mydef}[thm]{Definition}

\setlength\parindent{0pt}

% Def new command for "less than or equal" relation
\newcommand{\ineq}{%
	\mathrel{\mkern1mu\underline{\mkern-1mu\in\mkern-1mu}\mkern1mu}}

\newcommand{\opair}[2]{\langle{#1}, {#2} \rangle}
\newcommand{\myset}[2]{\{{#1} \mid {#2}\}}

\begin{document}

\setcounter{section}{2}
\section{Relations and Functions}
\subsection{Ordered Pairs}

\begin{mydef}
	$\opair{x}{y}$ is defined to be ${{x},{x,y}}$.
\end{mydef}

\emph{Cartesian product} $A \times B$ of $A$ and $B$:
\begin{equation}
	A \times B = \myset{\opair{x}{y}}{x \in A \land y \in B}
\end{equation}

\begin{enumerate}[label=\arabic*.]
	\item $A \times (B \cup C) = (A \times B) \cup (A \times C)$
	\item If $A \times B = A \times C$ and $A \ne \emptyset$, then $B = C$
	\item $A \times \bigcup\mathscr{B} = \bigcup\myset{A \times X}{X \in mathscr{B}}$
\end{enumerate}

\subsection{Relations}
\begin{mydef}
	A \emph{realtion} is a set of ordered pairs.
\end{mydef}

\begin{mydef}
	Define the \emph{domain} of $R$ (dom $R$), the \emph{range} of $R$ (ran
	$R$), and the \emph{field} of $R$ (fld $R$) by
	\begin{align*}
		x \in dom(R) &\iff \exists y \opair{x}{y} \in R \\
		x \in ran(R) &\iff \exists t \opair{t}{x} \in R \\
		fld(R) &= dom(R) \cup ran(R)
	\end{align*}
\end{mydef}

\begin{enumerate}
	\item $dom(\bigcup \mathscr{A}) = \bigcup\myset{dom($R$)}{R \in
	\mathscr{A}}
	\item $ran(\bigcup \mathscr{A}) = \bigcup\myset{ran($R$)}{R \in
	\mathscr{A}}
\end{enumerate}

\subsection{Functions}

\begin{mydef}
	A \emph{function} is a relation $F$ such that for each $x$ in $dom(F)$
	there is only one $y$ such that $xFy$.
\end{mydef}

\begin{mydef}
	\begin{enumerate}[label=(\alph*)]
		\item The \emph{inverse} of $F$ is the set
			\begin{equation*}
				F^{-1} = \myset{\opair{u}{v}}{vFu}
			\end{equation*}
		\item The \emph{composition} of $F$ and $G$ is the set
			\begin{equation*}
				F \circ G = \myset{\opair{u}{v}}{\exists t (uGt \land tFv)}
			\end{equation*}
		\item The \emph{restriction} of $F$ to $A$ is the set
			\begin{equation*}
				F \restriction A = \myset{\opair{u}{v}}{uFv \land u \in A}
			\end{equation*}
		\item The \emph{image} of $A$ under $F$ is the set
			\begin{equation*}
				F[A] = \myset{v}{(\exists u \in A)uFv}
			\end{equation*}
	\end{enumerate}
\end{mydef}

Equality of functions
Assume that $F$ and $G$ are functions, $dom(F) = dom(G) = D$, and $\forall x
\in D (F(x) = G(x))$. Then $F = G$.
\end{document}
